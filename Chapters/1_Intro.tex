\documentclass[../main.tex]{subfiles}
\begin{document}
\setchapterstyle{kao}
\setchapterpreamble[u]{\margintoc}
\setchapterimage[6.5cm]{Images/Leonardo_da_Vinci_-_Uomo_vitruviano.jpg}
\chapter[Introduction]{Introduction\footnotemark[0]}
\labch{intro}
\section{Symmetries}
\subsection{Classical Mechanics}
In classical mechanics, we have systems with a finite number of degrees of freedom: we have coordinates $q_i$ and momenta $p_i$. Given a Hamiltonian $H(q_i,p_i)$, we know that the equations of motion are given by the \href{https://en.wikipedia.org/wiki/William_Rowan_Hamilton}{Hamilton}-\href{https://en.wikipedia.org/wiki/William_Rowan_Hamilton}{Jacobi} equations:
\[
\left\{
\begin{aligned}
&\Dot{q}_i=\frac{\partial H}{\partial p_i}\\
&\Dot{p}_i=-\frac{\partial H}{\partial q_i}
\end{aligned}
\right.
\]
We introduce the \href{https://en.wikipedia.org/wiki/Siméon-Denis_Poisson}{Poisson} brackets:
\[
\{f(q,p),g(q,p)\}=\sum_i\left[\frac{\partial f}{\partial q_i}\frac{\partial g}{\partial p_i}-\frac{\partial f}{\partial p_i}\frac{\partial g}{\partial q_i}\right]
\]
Using them, it is possible to rewrite the equations of motion in the following form:
\[
\left\{
\begin{aligned}
&\Dot{q}_i=\{q_i,H\}\\
&\Dot{p}_i=\{p_i,H\}
\end{aligned}
\right.
\quad
\left\{
\begin{aligned}
&\{q_i,q_j\}=\{p_i,p_j\}=0\\
&\{q_i,p_j\}=\delta_{ij}
\end{aligned}
\right.
\]
Consider now a generic transformation $f$ which could in principle depends on the coordinates $q_i$, the momenta $p_i$ and the time $t$:
\[
\frac{d}{dt}f(p,q,t)=\frac{\partial f}{\partial t}+\sum_i\left[\frac{\partial f}{\partial q_i}\Dot{q}_i+\frac{\partial f}{\partial p_i}\Dot{p}_i\right]=\frac{\partial f}{\partial t}+\sum_i\left[\frac{\partial f}{\partial q_i}\frac{\partial H}{\partial p_i}-\frac{\partial f}{\partial p_i}\frac{\partial H}{\partial q_i}\right]=\frac{\partial f}{\partial t}+\{f,H\}
\]
Moreover, we can be more specific and focus on transformations of the type:
\[
\left\{
\begin{aligned}
&q_i\to\Bar{q}_i(q,p)\\
&p_i\to\Bar{p}_i(q,p)
\end{aligned}
\right.
\]
This transformation is called \textbf{canonical} if the new variables $\Bar{q}_i$ and $\Bar{p}_i$ satisfy the same rules given by the Poisson brackets of $q_i$ and $p_i$. If this is true, then the equations of motion remain unchanged. The Poisson brackets are also conserved under canonical transformations. We need to distinguish between \textbf{active} and \textbf{passive} transformations:
\begin{itemize}
    \item \underline{Active transformations}: $(q,p)$ and $(\Bar{q},\Bar{p})$ are different points in phase space, $f(q,p)\neq f(\Bar{q},\Bar{p})$
    \item \underline{Passive transformations}: $(q,p)$ and $(\Bar{q},\Bar{p})$ are the same point in phase space, $f(q,p)=\Bar{f}(\Bar{q},\Bar{p})$.
\end{itemize}
\textbf{Symmetries} are defined as active regular transformations which leave the Hamiltonian (and therefore the equations of motion) invariant:\\
$H(q,p)=H(\Bar{q},\Bar{p})$. Among the canonical transformations, we find the point transformations which are function only of $(q,t): q_i\to\Bar{q}_i(q,t)$. From this, we can introduce the conjugated momenta $p_i$, defined as $p_i:=\partial L(q,\Dot{q})/\partial q_i$. It is also possible to consider infinitesimal canonical transformations, given by:
\[
\left\{
\begin{aligned}
&q_i\to q_i+\varepsilon\frac{\partial g}{\partial p_i}=q_i+\varepsilon\{q_i,g\}=q_i(\varepsilon)\\
&p_i\to p_i+\varepsilon\frac{\partial g}{\partial q_i}=p_i+\varepsilon\{p_i,g\}=p_i(\varepsilon)
\end{aligned}
\right.
\]
$g(q,p)$ is called the generator of the infinitesimal canonical transformation. For finite transformations, we find something similar to the equations of motion:
\[
\frac{d}{d\varepsilon}q_i(\varepsilon)=\{q_i,g\} \quad \frac{d}{d\varepsilon}p_i(\varepsilon)=\{p_i,g\}
\]
Consider now a symmetry generated by $g(q,p)$, it is possible to show that:
\begin{enumerate}
    \item $g(q,p)$ is a constant of motion, $dg(q,p)/dt=\{g,H\}=0$. This is because if we compute the variation of $H$ we find:\marginnote{Since it is a symmetry the Hamiltonian is invariant.}
    \[
    0=\delta H=\sum_i\left[\frac{\partial H}{\partial q_i}\delta q_i+\frac{\partial H}{\partial p_i}\delta p_i\right]=\varepsilon\sum_i\left[\frac{\partial H}{\partial q_i}\frac{\partial g}{\partial p_i}+\frac{\partial H}{\partial p_i}\frac{\partial g}{\partial q_i}\right]=\varepsilon\{H,g\}
    \]
    \item if $(q,p)$ is a solution of the equations of motion, then also $(\Bar{q},\Bar{p})$ is a solution of the equations of motion.
\end{enumerate}
\subsection{Classical Field Theory}
To start our discussion in field theory, as a first step we discretize the space in a finite volume, i.e. $\phi(\Vec{x},t)\to\phi(\Vec{x}_m,t)=\phi_m(t)$. We can also define the conjugated momenta as $\pi_m(t)=\partial L(\phi,\Dot{\phi})/\partial\Dot{\phi}_m$ under the requirements of having:
\[
\left\{
\begin{aligned}
&\{\phi(\Vec{x}_m,t),\phi(\Vec{x}_n,t)\}=\{\pi_m(t),\pi_n(t)\}=0 \quad &&\{\phi_m(t),\pi_n(t)\}=\delta_{mn}\\
&\{\phi(\Vec{x},t),\phi(\Vec{y},t)\}=\{\pi(\Vec{x},t),\pi(\Vec{y},t)\}=0 \quad &&\{\phi(\Vec{x},t),\pi(\Vec{y},t)\}=\delta^3(\Vec{x}-\Vec{y})
\end{aligned}
\right.
\]
Also in this case, we can define infinitesimal point transformations:
\[
\left\{
\begin{aligned}
&\phi_i(x)\to\phi_i(x)+i\varepsilon F_i(\phi)=\Bar{\phi}_i(x)\\
&\pi_i(x)\to\frac{\partial L(\Bar{\phi},\Dot{\Bar{\phi}})}{\partial\Dot{\phi}}=\Bar{\pi}_i(x)
\end{aligned}
\right.
\]
In field theory, symmetries are point transformations which leave the action $S[\phi]$ (and therefore the equations of motion) invariant:
\[
S[\phi]=\int dt L(\phi,\Dot{\phi})=\int d^4x\pazocal{L}(\phi,\Dots{\phi}) \quad \delta S=S[\phi+\delta\phi]-S[\phi]=0
\]
We know from \href{https://en.wikipedia.org/wiki/Emmy_Noether}{N\"other}'s theorem that for every symmetry there is a conserved current $J^\mu$ associated and a conserved charge $Q=\int d^3xJ^0(\Vec{x},t)$:
\[
\partial_\mu J^\mu(x)=0\Rightarrow \frac{d}{dt}Q=\int d^3x\frac{\partial}{\partial t}J^0(\Vec{x},t)=\int d^3x\Vec{\nabla}\cdot\Vec{J}=\lim_{V\to\infty}\int_{\partial V}dS\hat{n}\cdot\Vec{J}=0
\]
If $\varepsilon$ is a constant it is easy to see that $\delta S=0$, now we look at the case of $\varepsilon=\varepsilon(x)$:
\[
\delta S=\int d^4x\partial_\mu\varepsilon(x)J^\mu(x)\underset{\mathclap{\tikz \node {$\uparrow$} node [below=1ex] {\footnotesize by parts };}}{=}-\int d^4x\partial_\mu J^\mu(x)\varepsilon(x)=0 \quad \text{if}\;\partial_\mu J^\mu=0
\]
If the Lagrangian density itself $\pazocal{L}$ is invariant, then we have:
\[
\delta S=\int d^4x\left[\frac{\delta\pazocal{L}}{\delta\phi(x)}i\varepsilon F_i(\phi)+\frac{\delta\pazocal{L}}{\delta(\partial_\mu\phi)}\partial_\mu(i\varepsilon F_i(\phi))\right]=0
\]
Being $\varepsilon=$ constant and $\pazocal{L}$ invariant, it follows that:
\[
\frac{\delta\pazocal{L}}{\delta\phi}F_i(\phi)+\frac{\delta\pazocal{L}}{\delta(\partial_\mu\phi)}\partial_\mu F_i(\phi)=0
\]
It is possible to express the current $J^\mu$ as:
\[
J^\mu=\frac{\delta\pazocal{L}}{\delta(\partial_\mu\phi_i(x))}iF_i(\phi) \quad J^0=\frac{\delta\pazocal{L}}{\delta\Dots{\phi}_i(x)}iF_i(\phi)=i\pi_iF_i(\phi)
\]
The conserved charge $Q$ acts as the generator of the symmetry. We can observe that:
\begin{align*}
\{\phi_i(x),Q\}&=\int d^3y\{\phi_i(\Vec{x},t),J^0(\Vec{y},t)\}=i\int d^3y\{\phi_i(\Vec{x},t),\pi_k(\Vec{y},t)F_k(\phi(\Vec{y},t))\}\\
&=i\int d^3y\{\phi_i(\Vec{x},t),\pi_k(\Vec{y},t)\}F_k(\phi(\Vec{y},t))=i\int d^3y\delta_{ik}\delta^3(\Vec{x}-\Vec{y})F_k(\phi(\Vec{y},t))\\
&=iF_i(\phi(\Vec{x},t))=\frac{\partial\phi_i(x)}{\partial\varepsilon}
\end{align*}
Symmetry transformations have the mathematical structure of a \textbf{group}\cite{gruppi}, which is nothing but a set of elements with an internal operation satisfying the following properties:
\begin{enumerate}
    \item \underline{Associativity}: $g_1,g_2,g_3\in$\,G, $g_1\cdot(g_2\cdot g_3)=(g_1\cdot g_2)\cdot g_3$ 
    \item \underline{Identity element}: $\exists\,e\in$\,G such that $\forall g\in$\,G $g\cdot e=e\cdot g=g$
    \item \underline{Inverse element}: $\forall g\in$\,G$\exists\,g^{-1}$ such that $g\cdot g^{-1}=g^{-1}\cdot g=e$
\end{enumerate}
We are mainly interested in \href{https://en.wikipedia.org/wiki/Sophus_Lie}{Lie} groups, i.e. a group with an infinite number of elements which is also a differential manifold. The tangent space of a Lie group in any given point is a Lie \textbf{algebra}. An algebra is a vector space with an associated bilinear product indicated by $[\dots,\dots]$. It is bilinear if:
\begin{itemize}
    \item $[A+B,C]=[A,C]+[B,C]$
    \item $[A,B+C]=[A,C]+[A,B]$
    \item $[aA,bB]=ab[A,B] \; \forall a,b$ scalar
\end{itemize}
To define the Lie algebra, we need an additional property which is the Jacobi identity:
\[
[A,[B,C]]+[B,[C,A]]+[C,[A,B]]=0
\]
A bilinear product satisfying these four properties is called a \textbf{Lie bracket}.\\
We are interested in a particular class of Lie groups, which consists of linear transformations on a vector space $V$, that can be both finite- and infinite-dimensional. If $V$ is finite-dimensional, then it is isomorphic to $\mathbb{R}^n$ or $\mathbb{C}^n$: we can treat vectors as column of numbers and linear operators as matrices. These are called \textbf{linear Lie groups} and for linear Lie groups, the generators are the elements of the algebra. We focus our attention on compact, connected linear Lie groups, which can be divided into classical groups (3 families) and exceptional groups.
\begin{itemize}
    \item \underline{Classical groups}:
    \begin{enumerate}
        \item Special unitary $n\times n$ matrices, denoted with SU$(n)$. It preserves the complex scalar product $w\cdot v=w_a^*\cdot v_a$. Special means that if $U\in$\,SU$(n)$, then $\det U=1$. SU$(n)$ has $n^2-1$ generators, which are Hermitian matrices.
        \item Special orthogonal $n\times n$ matrices, denoted with SO$(n)$. It preserves the real scalar product $w\cdot v=w_a\cdot v_a$. SO$(n)$ has $n(n-1)/2$ generators which are anti-symmetric matrices.
        \item Unitary symplectic $n\times n$ matrices, denoted with Sp$(n)$. It preserves $w_a E_{ab}v_b$, where $E_{ab}$ is:
        \[
        E_{ab}=\left(\begin{array}{cc}
            0 & \mathbb{1} \\
            -\mathbb{1} & 0
        \end{array}\right)
        \]
        where $\mathbb{1}$ is the $n/2$-dimensional identity. Sp$(n)$ has $n(n+1)/2$ generators.
    \end{enumerate}
    \item\underline{Exceptional groups}: G$_2$, F$_4$, E$_6$, E$_7$, E$_8$, which are useful for grand unified field theory.\marginnote{I have no idea what they represent or why he thought it was useful to mention them in a course about electroweak interactions in which we have seen nor electroweak nor the interactions.}
\end{itemize}
An algebra of a group is characterized by the commutation rules of the generators of the group: $[T^a,T^b]=if^{abc}T^c$, where $f^{abc}$ is the so-called \textbf{structure constant}. If $f^{abc}=0$, then the generators commute and the group is called \textbf{Abelian} (U(1) is an Abelian group for example). In an Abelian Lie group, we can reach any point of the manifold in the following way:
\[
g(\alpha)=\lim_{N\to\infty}\left(1+\frac{i}{N}\alpha T\right)^N=\exp{i\alpha T}
\]
where $T$ is the generator, $N$ is the number of steps and $\alpha$ is a real parameter. Every element of the group G can be written in this way: we realized an \textbf{exponential map} connecting the generators (i.e. elements of the algebra) to elements of the group. For non-Abelian groups, this is generally not true: we cannot reach every element of the group, but there are some exceptions. The exponential map is \textbf{surjective} when the group is compact and connected.
\hline
As an example, consider a classical theory of $n$ real scalar fields $\phi_a$:
\[
\pazocal{L}=\frac{1}{2}\partial_\mu\phi_aK_{ab}\partial^\mu\phi_b-\frac{1}{2}\phi_a\phi_b(M^2)_{ab}
\]
where both $K_{ab}$ and $(M^2)_{ab}$ are symmetric, real matrices. To simplify our life, we can perform a rotation $R\in$\,SO$(n)$:
\[
\phi_a\to R_{ab}\phi_b \quad (R^TKR)_{ab}=(K_{diag})_{ab}=K_a\delta_{ab}
\]
We can also re-scale the field, so that $\phi_a\to\phi_a/\sqrt{K_a}$ and $K_{ab}=\delta_{ab}$. Another possible choice for the rotation $R$ is to choose it so that we diagonalize the matrix $M^2$:
\[
\phi_a\to R'_{ab}\phi_b \quad (R'^TM^2R)_{ab}=(M^2)_a\delta_{ab}
\]
The Lagrangian now becomes:
\[
\pazocal{L}=\frac{1}{2}(\partial_\mu\phi_a)^2-\frac{1}{2}\phi_a^2M_a^2
\]
In this case, there is no global invariance for a generic $M_a^2$. If $M_a=M_b\;\forall a,b$ i.e. $M_{ab}^2=m^2\delta_{ab}$ then we have a SO$(n)$ invariance. We have seen that SO$(n)$ has $n(n-1)/2$ generators $T^A$, the rotation we just performed can be expressed in terms of them:
\[
\phi_a\to R_{ab}\phi_b\simeq\{\delta_{ab}+iT_{ab}^A\alpha^A+\pazocal{O}(\alpha^2)\}\phi_b=\phi_a+iT_{ab}^A\alpha^A\phi_b+\pazocal{O}(\alpha^2)
\]
$(iT^a)$ must be real and anti-symmetric, now we take a look at the variation of the Lagrangian:
\[
\delta\pazocal{L}=\partial_\mu\phi_a\partial_\mu\alpha^AiT^A_{ab}\phi_b=\partial_\mu\alpha^AJ^{\mu,A}(x)
\]
where we highlighted the current $J^{\mu,A}=\partial_\mu\phi_aiT_{ab}^A\phi_b$. If we consider for example the case with $n=2$, we can write the generators of the rotation as:
\[
iT_{ab}^A=\left(\begin{array}{cc}
    0 & -1 \\
    +1 & 0
\end{array}\right)
\]
This corresponds to an anti-clockwise rotation, which gives us the current $J^\mu=\phi_1\partial_\mu\phi_2-\phi_2\partial_\mu\phi_1$. We move now to the case of $n$ complex scalar field, for which we have a Lagrangian of the form:
\[
\pazocal{L}=\partial_\mu\phi_a^*K_{ab}\phi_b-\phi_a^*(M^2)_{ab}\phi_b
\]
where $K_{ab}$ and $(M^2)_{ab}$ are hermitian matrices. Being them hermitian, we no longer want an orthogonal transformation but we are looking for a \textbf{unitary} one. We therefore perform the following simplifications:
\[
\left\{
\begin{aligned}
&\phi_a\to U_{ab}\phi_b \;\text{such that}\;K\to K_{diag}\\
&\phi_a\to\frac{1}{\sqrt{K_a}}\phi_a\\
&\phi_a\to U_{ab}'\phi:b \;\text{such that}\;(M^2)_{ab}=m_a^2\delta_{ab}
\end{aligned}
\right.
\]
In this way, the Lagrangian becomes:
\[
\pazocal{L}=|\partial_\mu\phi_a|^2-\phi_a^*\phi_am_a^2
\]
For generic masses, we have a [U$(1)$]$^n$ symmetry, while for $M^2=m^2\mathbb{1}$ the symmetry is U$(n)\simeq$SU$(n)\times$U$(1)$.
\end{document}